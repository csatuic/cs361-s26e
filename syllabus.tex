% !TEX program = xelatex
\documentclass[11pt]{article}

% -----------------------------
% Page layout
% -----------------------------
\usepackage[margin=1in]{geometry}

% -----------------------------
% Font: Calibri (requires XeLaTeX or LuaLaTeX)
% -----------------------------
\usepackage{fontspec}
\IfFontExistsTF{Calibri}{
  \setmainfont{Calibri}
}{
  \setmainfont{Arial}
}

% -----------------------------
% Common utilities
% -----------------------------
\usepackage{microtype}
\usepackage{xcolor}
\usepackage{hyperref}
\usepackage{enumitem}
\usepackage{longtable}
\usepackage{array}
\usepackage{booktabs}

% Spacing
\usepackage{setspace}

% Headings with gray box
\usepackage[explicit]{titlesec}
\usepackage{tikz}

% -----------------------------
% Hyperlinks
% -----------------------------
\hypersetup{
  colorlinks=true,
  linkcolor=blue,
  urlcolor=blue,
  citecolor=blue
}

% Word-like paragraph behavior (tighter)
\setlength{\parindent}{0pt}
\setlength{\parskip}{0.18\baselineskip} % was 0.55
\setstretch{1.03}                      % was 1.12
\raggedbottom

% -----------------------------
% Lists (Word-like indentation and tight spacing)
% -----------------------------
\setlist[itemize]{leftmargin=1.25em, itemsep=0.15em, topsep=0.15em, parsep=0pt, partopsep=0pt}
\setlist[enumerate]{leftmargin=1.25em, itemsep=0.15em, topsep=0.15em, parsep=0pt, partopsep=0pt}

% -----------------------------
% Gray boxed section heading (Word-like)
% -----------------------------
\definecolor{uiclightgray}{RGB}{200,200,200}

\titleformat{\section}
  {\Large\bfseries\sffamily}
  {}
  {0em}
  {%
    \begin{tikzpicture}[baseline=(title.base)]
      \node[
        inner sep=8pt,
        fill=uiclightgray,
        draw=black,
        line width=0.8pt,
        text width=\dimexpr\textwidth-16pt\relax,
        align=left,
        anchor=west
      ] (title) {#1};
    \end{tikzpicture}%
  }
  []

% Control vertical spacing around section bars
\titlespacing*{\section}{0pt}{0.9\baselineskip}{0.6\baselineskip}

% Subsections: compact and bold like the Word template
\titleformat{\subsection}{\bfseries}{\thesubsection}{0.5em}{#1}
\titlespacing*{\subsection}{0pt}{0.8\baselineskip}{0.35\baselineskip}

% -----------------------------
% Table helpers
% -----------------------------
\newcolumntype{L}[1]{>{\raggedright\arraybackslash}p{#1}}

% Longtable spacing and padding similar to Word
\setlength{\LTpre}{0pt}
\setlength{\LTpost}{0pt}
\renewcommand{\arraystretch}{1.02}
\setlength{\tabcolsep}{6pt}

% -----------------------------
% Editable course metadata
% -----------------------------
\newcommand{\collegeName}{COLLEGE OF ENGINEERING, UIC}

\newcommand{\courseRubric}{CS}
\newcommand{\courseNumber}{361}
\newcommand{\courseTitle}{Systems Programming}
\newcommand{\creditHours}{4}

\newcommand{\termSemesterYear}{Spring 2026}

\newcommand{\instructorName}{Jakob Eriksson}
\newcommand{\instructorEmail}{jakob@uic.edu}
\newcommand{\officeHours}{Thursdays noon-1pm, Fridays noon-1pm}
\newcommand{\officeHoursLocation}{Thursdays: The Stack, Fridays: https://uic.zoom.us/my/prof.jakob}

% \newcommand{\coInstructorName}{Teaching Assistant or Co-Instructor Name}
% \newcommand{\coInstructorEmail}{ta@uic.edu}
% \newcommand{\sectionDesignations}{Section designation(s)}
% \newcommand{\coOfficeHours}{Wed }
% \newcommand{\coOfficeHoursLocation}{Office location or Zoom link}

\newcommand{\blackboardLinkText}{Piazza}
\newcommand{\blackboardLinkUrl}{https://piazza.com}

\newcommand{\catalogDescription}{%
This course is a study of computer systems emphasizing their impact on application level programming. 
By the end of the class, you will have a good low-level understanding of how applications
are built from source code, how processes are created from executables, and the environment that
processes run in, all primarily from a Unix perspective. In addition, you will have a good familiarity
with the most important tools that a systems programmer uses, including build tools, debuggers,
profilers and tracers.

\textbf{Prerequisite:} CS261
}

\newcommand{\growthMindsetStatement}{%
Course materials and assignments can be complex and challenging, but they are crucial
to your intellectual and personal growth and development. There are times you may need
extra help. Students who attend class consistently, complete all assignments, thoughtfully
engage with feedback on work, develop good study strategies, visit the tutoring center,
and contact faculty when struggling can develop a thorough understanding of the course
material and ultimately succeed in the course.
}

\newcommand{\courseGoalsLearningOutcomes}{%
\begin{itemize}
\item  Analyze a complex computing problem and apply principles of computing and other relevant disciplines to identify solutions.
\item Design, implement, and evaluate a computing-based solution to meet a given set of computing requirements in the context of the program's discipline.
%\item Communicate effectively in a variety of professional contexts.
%\item Recognize professional responsibilities and make informed judgments in computing practice based on legal and ethical principles.
%\item Function effectively as a member or leader of a team engaged in activities appropriate to the program’s discipline.
\end{itemize}

\textbf{Course requirement status:} Required for Computer Science majors.\\
%\textbf{Topics overview:} [Brief list of topics to be covered.]
}

\newcommand{\courseMaterials}{%
{\bf Textbook:} Computer Systems a Programmer's Perspective, by Bryant and O'Hallaron. 
\\
{\bf Technology:} This course requires the use of a standard portable computer running Linux, Mac OS or Windows.
You must have an ssh client software, and be familiar with its use. 
I recommend using Visual Studio Code with the Remote SSH plug-in for editing files on department servers.
}

% \newcommand{\copyrightStatement}{%
% Please protect the copyright integrity of all course materials and content.
% Please do not upload course materials not created by you onto third-party websites
% or share content with anyone not enrolled in our course.
% }

\newcommand{\gradingPolicy}{

Your final letter grade is determined based on your course percentage, which has the following components:

\begin{itemize}
  \item 10 Lab completion credits (weight 1 each)
  \item 10 Post-assignment paper quizzes (weight 4 each)
  \item Midterm paper exam (weight 20)
  \item Final paper exam (weight 30)
  \item Sum of weights: 100
\end{itemize}

The course percentage is the weighted average percentage of the final and the other 21 components above, with {\bf one major modification}: 
with the exception of the final exam, if ignoring a component would improve your course percentage, then it is not counted. 

The purpose of the quiz vs. assignment-based grading policy is to emphasize individual learning, rather than skill in completing assignments or AI prompting. 
The purpose of the course percentage computation is to provide maximum flexibility. 

For example, if you score 100\% on the final, then no other components are counted, and your course percentage is 100\%. 
If you miss the final, and the midterm, but score 100\% on all the labs and quizzes, then your course percentage is $50/(50+30)$, as the
midterm is ignored, but the final is not.

The course score cut-offs for the final letter grades (A, B, C, D, F) will be set after the final exam results are in.
}

\newcommand{\lateWorkPolicy}{%
There is no explicit accomodation for late work in this course, see grading policy.
}

\newcommand{\attendanceParticipationPolicy}{%
Attendance is not taken nor required in this course. 
}

\newcommand{\academicIntegrityPolicy}{%
You may use any tools or methods you see fit to learn the materials of this course and to 
complete assignments. This includes enlisting the help of classmates, instructors and all AI chatbots.

However, you will be evaluated on your own knowledge, through paper-based quizzes and exams. 

Please review the UIC Student Disciplinary Policy for additional information.
}

\newcommand{\emailExpectations}{%
Please use Piazza for all class-related communication.  For example,
questions about assignments, missing quizzes, exam results, study technique, etc are all suitable for Piazza. 
}

\newcommand{\scheduleDisclaimer}{%
This syllabus is intended to give the student guidance on what may be covered during the semester
and will be followed as closely as possible. However, as the instructor, I reserve the right to
modify, supplement, and make changes as course needs arise. I will communicate such changes in
advance through in-class announcements on Piazza.
}

\newcommand{\disabilityAccommodation}{%
UIC is committed to full inclusion and participation of people with disabilities in all aspects of university life.
If you face or anticipate disability-related barriers while at UIC, please connect with the Disability Resource Center (DRC)
at \url{https://drc.uic.edu}, via email at \href{mailto:drc@uic.edu}{drc@uic.edu}, or call (312) 413-2183 to create a plan for
reasonable accommodations. To receive accommodations, you will need to disclose the disability to the DRC, complete an interactive
registration process with the DRC, and provide me with a Letter of Accommodation (LOA). Upon receipt of an LOA, I will gladly work
with you and the DRC to implement approved accommodations.
}

\newcommand{\religiousAccommodation}{%
Following campus policy, if you wish to observe religious holidays, you must notify me by the tenth day of the semester.
If the religious holiday is observed on or before the tenth day of the semester, you must notify me at least five days
before you will be absent. Please submit this form by email with the subject heading: ``YOUR NAME: Requesting Religious Accommodation.''
}

% \newcommand{\inclusiveCommunity}{%
% UIC values diversity and inclusion. Regardless of age, disability, ethnicity, race, gender, gender identity, sexual orientation,
% socioeconomic status, geographic background, religion, political ideology, language, or culture, we expect all members of this class
% to contribute to a respectful, welcoming, and inclusive environment for every other member of our class. If aspects of this course
% result in barriers to your inclusion, engagement, accurate assessment, or achievement, please notify me as soon as possible.
% }

% \newcommand{\namePronounUse}{%
% If your name does not match the name on my class roster, please let me know as soon as possible.
% My pronouns are [she/her; he/him; they/them]. I welcome your pronouns if you would like to share them with me.
% For more information about pronouns, see: \url{https://www.mypronouns.org/what-and-why}.
% }

% \newcommand{\communityAgreement}{%
% \begin{itemize}
%   \item Be present by turning off cell phones and removing yourself from other distractions.
%   \item Be respectful of the learning space and community.
%   \item Use preferred names and gender pronouns.
%   \item Assume goodwill in all interactions, even in disagreement.
%   \item Facilitate dialogue and value the free and safe exchange of ideas.
%   \item Try not to make assumptions, have an open mind, seek to understand, and not judge.
%   \item Debate the concepts, not the person.
%   \item Be willing to work together and share helpful study strategies.
%   \item Be mindful of one another's privacy and do not invite outsiders into our classroom.
% \end{itemize}
% }

% \newcommand{\contentNotices}{%
% Our classroom provides an open space for a critical and civil exchange of ideas, inclusive of a variety of perspectives and positions.
% Some readings and other content may expose you to ideas, subjects, or views that may challenge you, cause you discomfort, or recall past
% negative experiences or traumas. I intend to discuss all subjects with dignity and humanity, as well as with rigor and respect for scholarly inquiry.
% If you would like me to be aware of a specific topic of concern, please email or visit my Student Drop-In Hours.
% }

% \newcommand{\resourcesSection}{%
% We all need the help and the support of our UIC community. Please visit my office hours for course consultation and other academic or research topics.
% For additional assistance, please contact your assigned college advisor and visit the support services available to all UIC students.

% \subsection*{Academic Success}
% \begin{itemize}
%   \item UIC tutoring resources
%   \item College of Engineering tutoring program
%   \item Equity and Inclusion in Engineering Program
%   \item UIC Library and UIC Library Research Guides
%   \item Offices supporting the UIC undergraduate experience and academic programs
%   \item Student Guide for Information Technology
% \end{itemize}

% \subsection*{Wellness}
% \begin{itemize}
%   \item Counseling Services: \url{https://counseling.uic.edu/}
%   \item Access U\&I Care Program for assistance with personal hardships
%   \item Campus Advocacy Network (Title IX): \href{mailto:TitleIX@uic.edu}{TitleIX@uic.edu} \quad and \quad \url{http://can.uic.edu/}
% \end{itemize}

% \subsection*{Safety}
% \begin{itemize}
%   \item UIC Safe App (please download for your safety)
%   \item UIC Safety Tips and Resources
%   \item Night Ride
%   \item Emergency: Dial 5-5555 from a campus phone, or (312) 355-5555 from a cell phone
% \end{itemize}
% }

\begin{document}

% -----------------------------
% Centered header block
% -----------------------------
\begin{center}
  {\LARGE\bfseries \collegeName}\par
  \vspace{0.75em}
  {\Large\bfseries
    \courseRubric\ \courseNumber\quad
    \courseTitle\quad
    (\creditHours\ Credit Hours)
  }\par
  \vspace{0.5em}
  {\large \termSemesterYear}\par
\end{center}

\vspace{0.9\baselineskip}

% -----------------------------
% Instructor and course details
% -----------------------------
\section*{Instructor \& Course Details}

\noindent
\begin{tabular}{@{}p{0.28\linewidth}p{0.70\linewidth}@{}}
\textbf{Days/Times \& Location:} & TR 3:30-4:45 in CDRLC 1426\\
\textbf{Instructor:} & \instructorName \\
\textbf{Email:} & \href{mailto:\instructorEmail}{\instructorEmail} \\
\textbf{Drop-In Office Hours:} & \officeHours \\
\textbf{Drop-In Hours Location:} & \officeHoursLocation \\

\end{tabular}

% \vspace{0.6\baselineskip}

% % \noindent
% % \begin{tabular}{@{}p{0.28\linewidth}p{0.70\linewidth}@{}}
% % \textbf{Co-Instructor/TA:} & \coInstructorName \\
% % \textbf{Email:} & \href{mailto:\coInstructorEmail}{\coInstructorEmail} \\
% % \textbf{Section designation(s):} & \sectionDesignations \\
% % \textbf{Drop-In Office Hours:} & \coOfficeHours \\
% % \textbf{Drop-In Hours Location:} & \coOfficeHoursLocation \\
% % \end{tabular}

% \vspace{0.6\baselineskip}

\noindent
\begin{tabular}{@{}p{0.28\linewidth}p{0.70\linewidth}@{}}
\textbf{Course site:} & \href{\blackboardLinkUrl}{\blackboardLinkText} \\
\end{tabular}

Students are expected to log into the course site regularly to learn about any developments related to the course,
and communicate with instructors and classmates.


% -----------------------------
% Course information
% -----------------------------
\section*{Course Information}

\subsection*{Catalog Course Description and Prerequisite/Corequisite Statement}
\catalogDescription

% \subsection*{Growth Mindset}
% \growthMindsetStatement

\subsection*{ABET Course Goals and Learning Outcomes}
\courseGoalsLearningOutcomes

\subsection*{Required Textbook \& Technology}
\courseMaterials

% \subsection*{Respect for Copyright}
% \copyrightStatement

% -----------------------------
% Course policies and classroom expectations
% -----------------------------
\section*{Course Policies \& Classroom Expectations}

\subsection*{Grading Policy and Point Breakdown}
\gradingPolicy

\subsection*{Policy for Missed or Late Work}
\lateWorkPolicy

\subsection*{Attendance and Participation Policy}
\attendanceParticipationPolicy

\subsection*{Academic Integrity}
\academicIntegrityPolicy

% \subsection*{Email Expectations}
% \emailExpectations

\newpage
% -----------------------------
% Course schedule
% -----------------------------
\section*{Course Schedule}

{
% --- Schedule table: inside margins + taller rows + no underfull hbox warnings ---
\renewcommand{\arraystretch}{1.25}
\setlength{\extrarowheight}{1.5pt}
\setlength{\tabcolsep}{4pt}


% Processes
%  Structure and Creation of Executables and Processes (building & linking & shared libraries & LD_PRELOAD)
%  Process Abstraction and System Call Interface (gdb & inline assembly (incl rdtscp) & signal delivery and handling)
%  Virtual memory

% I/O
%  File Descriptors - the Unix I/O abstraction (runCommand() & runInteractive(): fork, exec, dup2)
%  Network Programming with Sockets and Epoll (limerick server. "speakers" deliver words or syllables at a time.)

% Memory
%  Globals, Stack, Heap & Malloc (inspecting memory layout with gdb: address of globals, stack, heap. library globals, library locals, library heap variables. size of heap chunks.  )
%  Mmap and IPC (runInteractive, but comms through memory mapped file.)

% Concurrency
%  Multi-Threaded Programming Model (multi-threaded limerick server: subscriptions to atomically delivered limericks)
%  Race Conditions, Synchronization, Deadlocks (live streaming limerick server. tests check timing between syllables)

% Diagnostics Facilities
%  Debugging (basic debugger with ptrace api - catching system calls, printing register values)
%  Performance Diagnostics (working with time, perf, strace)


\begin{longtable}{@{}
  >{\raggedright\arraybackslash}p{\dimexpr 0.1\linewidth-2\tabcolsep\relax}
  >{\raggedright\arraybackslash}p{\dimexpr 0.3\linewidth-2\tabcolsep\relax}
  >{\raggedright\arraybackslash}p{\dimexpr 0.4\linewidth-2\tabcolsep\relax}
  >{\raggedright\arraybackslash}p{\dimexpr 0.2\linewidth-2\tabcolsep\relax}
@{}}
\toprule
\textbf{Week of} & \textbf{Topic(s) / In-Class Activities} & \textbf{Assignment due Tuesday 3:30pm} & \textbf{Reading}\\
\midrule
\endfirsthead

\toprule
\textbf{Week of} & \textbf{Topic(s) / In-Class Activities} & \textbf{Assignment due Tuesday 3:30pm} & \textbf{Reading}\\
\midrule
\endhead

\midrule
\multicolumn{4}{r}{(continued)}\\
\endfoot

\bottomrule
\endlastfoot

01/12 & Executables and Libraries & & CSaPP Ch 7 \\
01/19 & Processes and System Calls & hw1: linking and shared libraries & CSaPP Ch 8 \\
01/26 & File Descriptors and File I/O &  hw2: syscalls and signals & CSaPP Ch 10 \\
02/02 & Sockets and Network I/O & hw3: runCommand and runInteractive & CSaPP 11-11.4\\
02/09 & Adv. I/O & hw4: socket server with epoll & CSaPP 12.2 \\
02/16 & Memory Layout - Process and Data Structure & {\it no hw due} & CSaPP 9.1-9.2 \\
02/23 & Memory Management - Malloc and Mmap & hw5: learning about memory layout with gdb & CSaPP 9.8-9.9\\
03/02 & Adv. Memory Management and Common Problems & hw6: IPC over shared memory & CSaPP 9.11 \\
03/09 & MIDTERM EXAM on Tuesday, no lab & {\it no hw due} &  \\
03/16 & Programming with Threads & {\it no hw due} & 12.1,12.3\\
\hline
03/23 & \textit{SPRING BREAK} \\
\hline
03/30 & Race Conditions, Synchronization and Deadlock & hw7: multi-threaded socket server & CSaPP 12.4,12.6-12.7 \\
04/06 & Adv. Concurrency - atomics, lock-free operation & hw8: concurrency bug hunt & \\
04/13 & PTrace - the API for debugging & {\it no hw due} & \\
04/20 & Performance Monitoring with perf and strace & hw9: basic debugger with ptrace & \\
04/27 & Bonus Topic and Final Review & hw10: hands-on performance diagnostics \\
\hline
05/04 & \textit{FINAL EXAM WEEK} & Final Exam is on Friday May 7 at 1-3 PM \\
\end{longtable}
}

\subsection*{Disclaimer}
\scheduleDisclaimer

\newpage
% -----------------------------
% Accommodations
% -----------------------------
\section*{Accommodations}

\subsection*{Disability Accommodation Procedures}
\disabilityAccommodation

% \subsection*{Religious Accommodations}
% \religiousAccommodation

% % -----------------------------
% % Classroom environment
% % -----------------------------
% \section*{Classroom Environment}

% \subsection*{Inclusive Community}
% \inclusiveCommunity

% \subsection*{Name and Pronoun Use}
% \namePronounUse

% \subsection*{Community Agreement / Classroom Conduct Policy}
% \communityAgreement

% \subsection*{Content Notices}
% \contentNotices

% % -----------------------------
% % Resources
% % -----------------------------
% \section*{Resources: Academic Success, Wellness, and Safety}
% \resourcesSection

\end{document}